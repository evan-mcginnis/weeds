%%%%%%%%%%%%%%%%%%%%%%%%%%%%%%%%%%%%%%%%%%%%%%%
%%%     Declarations (skip to Begin Document, line 88, for parts you fill in)
%%%%%%%%%%%%%%%%%%%%%%%%%%%%%%%%%%%%%%%%%%%%%%%

%%\documentclass[10pt]{article}
%%\documentclass[10pt]{report}
\documentclass[letterpaper]{article}
\usepackage{geometry}
\usepackage{xcolor}
\usepackage{amsmath}
\usepackage[some]{background}
%\usepackage{lipsum}
%\usepackage{natbib}
\usepackage[backend=biber]{biblatex}  
%\usepackage{biblatex} 
\addbibresource{paperpile.bib}

% Tables
\usepackage{float}
\usepackage[utf8]{inputenc}
\usepackage{tabularx}
\usepackage{booktabs}
\usepackage{longtable}

\usepackage{diagbox} %table split headers
\usepackage{longtable}
\usepackage{array}
\usepackage{rotating}
\usepackage{eqparbox}
\usepackage{makecell, caption, booktabs}

%

\usepackage{geometry}  % Lots of layout options.  See http://en.wikibooks.org/wiki/LaTeX/Page_Layout
\geometry{letterpaper}  % ... or a4paper or a5paper or ... 
\usepackage{fullpage}  % somewhat standardized smaller margins (around an inch)
\usepackage{setspace}  % control line spacing in latex documents
\usepackage[parfill]{parskip}  % Activate to begin paragraphs with an empty line rather than an indent

\usepackage{amsmath,amssymb}  % latex math
\usepackage{empheq} % http://www.ctan.org/pkg/empheq
\usepackage{bm,upgreek}  % allows you to write bold greek letters (upper & lower case)

% for typsetting algorithm pseudocode see http://en.wikibooks.org/wiki/LaTeX/Algorithms_and_Pseudocode
\usepackage{algorithmic,algorithm}  

\usepackage{graphicx}  % inclusion of graphics; see: http://en.wikibooks.org/wiki/LaTeX/Importing_Graphics
% allow easy inclusion of .tif, .png graphics
\DeclareGraphicsRule{.tif}{png}{.png}{`convert #1 `dirname #1`/`basename #1 .tif`.png}

% \usepackage{subfigure}  % allows subfigures in figure
\usepackage{caption}
\usepackage{subcaption}

\usepackage{xspace}
\newcommand{\latex}{\LaTeX\xspace}

\usepackage{color}  % http://en.wikibooks.org/wiki/LaTeX/Colors

\long\def\todo#1{{\color{red}{\bf TODO: #1}}}

\long\def\ans#1{{\color{blue}{\em #1}}}
\long\def\ansnem#1{{\color{blue}#1}}
\long\def\boldred#1{{\color{red}{\bf #1}}}
\long\def\boldred#1{\textcolor{red}{\bf #1}}
\long\def\boldblue#1{\textcolor{blue}{\bf #1}}

% Useful package for syntax highlighting of specific code (such as python) -- see below
\usepackage{listings}  % http://en.wikibooks.org/wiki/LaTeX/Packages/Listings
\usepackage{textcomp}

% bem: insertions
\usepackage{listings}

%%% The following lines set up using the listings package
\renewcommand{\lstlistlistingname}{Code Listings}
\renewcommand{\lstlistingname}{Code Listing}

%%% Specific for python listings
\definecolor{gray}{gray}{0.5}
\definecolor{green}{rgb}{0,0.5,0}

\lstnewenvironment{python}[1][]{
\lstset{
language=python,
basicstyle=\footnotesize,  % could also use this -- a little larger \ttfamily\small\setstretch{1},
stringstyle=\color{red},
showstringspaces=false,
alsoletter={1234567890},
otherkeywords={\ , \}, \{},
keywordstyle=\color{blue},
emph={access,and,break,class,continue,def,del,elif ,else,%
except,exec,finally,for,from,global,if,import,in,i s,%
lambda,not,or,pass,print,raise,return,try,while},
emphstyle=\color{black}\bfseries,
emph={[2]True, False, None, self},
emphstyle=[2]\color{green},
emph={[3]from, import, as},
emphstyle=[3]\color{blue},
upquote=true,
morecomment=[s]{"""}{"""},
commentstyle=\color{gray}\slshape,
emph={[4]1, 2, 3, 4, 5, 6, 7, 8, 9, 0},
emphstyle=[4]\color{blue},
literate=*{:}{{\textcolor{blue}:}}{1}%
{=}{{\textcolor{blue}=}}{1}%
{-}{{\textcolor{blue}-}}{1}%
{+}{{\textcolor{blue}+}}{1}%
{*}{{\textcolor{blue}*}}{1}%
{!}{{\textcolor{blue}!}}{1}%
{(}{{\textcolor{blue}(}}{1}%
{)}{{\textcolor{blue})}}{1}%
{[}{{\textcolor{blue}[}}{1}%
{]}{{\textcolor{blue}]}}{1}%
{<}{{\textcolor{blue}<}}{1}%
{>}{{\textcolor{blue}>}}{1},%
%framexleftmargin=1mm, framextopmargin=1mm, frame=shadowbox, rulesepcolor=\color{blue},#1
framexleftmargin=1mm, framextopmargin=1mm, frame=single,#1
}}{}
%%% End python code listing definitions

\DeclareMathOperator{\diag}{diag}
\DeclareMathOperator{\cov}{cov}


%\bibliography{./paperpile.bib}
\author{Evan McGinnis}
\title{Monitor for XXX}



\definecolor{titlepagecolor}{cmyk}{1,.60,0,.40}

\DeclareFixedFont{\bigsf}{T1}{phv}{b}{n}{1.5cm}

\backgroundsetup{
scale=1,
angle=0,
opacity=1,
contents={\begin{tikzpicture}[remember picture,overlay]
 \path [fill=titlepagecolor] (-0.5\paperwidth,5) rectangle (0.5\paperwidth,10);  
\end{tikzpicture}}
}
\makeatletter                       
\def\printauthor{%                  
    {\large \@author}}              
\makeatother
\author{%
    Evan McGinnis \\
    Graduate Student, Biosystems Engineering Dept \\
    RNR 522
    \texttt{evanmc@email.arizona.edu}\vspace{40pt} \\
    }
\begin{document}
\begin{titlepage}
\BgThispage
\newgeometry{left=1cm,right=4cm}
\vspace*{1cm}
\noindent
%%\vspace*{0.4\textheight}
\textcolor{white}{\Huge\textbf{\textsf{Application of Vegetation Indices\\2D and 3D data}}}
\vspace*{2.5cm}\par
\noindent
\begin{minipage}{0.35\linewidth}
    \begin{flushright}
        \printauthor
    \end{flushright}
\end{minipage} \hspace{15pt}
%
\begin{minipage}{0.02\linewidth}
    \rule{1pt}{175pt}
\end{minipage} \hspace{-10pt}
%
\begin{minipage}{0.6\linewidth}
\vspace{5pt}
    \begin{abstract} 
This paper presents a breakdown of the analysis of the images taken along the Santa Cruz River, near the intersection of Silverbell Road and Sunset Blvd in Tucson, Arizona. Explored here are various approaches to image segmentation and subsequent estimates of vegetation coverage within the study area, estimates of vegetation volume, as well as a plan for longer term monitoring of the study area using these techniques. This study was conducted as part of the RNR 522 course. All code and data is available for review in GitHub and CyVerse, and readers should refer to the section of this document describing access.
    \end{abstract}
\end{minipage}
\end{titlepage}
\restoregeometry
%
% I N T R O D U C T I O N
%
\section{Introduction}
This results presented in this paper are just one snapshot of a longer effort to monitor the overall growth of vegetation in the study area, and to do so with a minimum of human input to the process. That it, the analysis is nearly completely automated and can be repeated over subsequent seasons.

%
% M E T H O D S
%
\section{Methods}
This study was conducted along the Santa Cruz River near the intersection of Silverbell Road and Sunset Blvd in Tucson, Arizona. Data was acquired with a DJI Phantom 4 with a stock RGB sensor capturing 12.4 Mp images. The images were captured between 09:21 on 10 November 2021 and 10:28 on 3 November 2021. Additional flight and capture details:
\begin{itemize}
\item Lens 8.8mm focal length
\item Exposure Mode: Program AE
\item ISO: 100
\item Location: 32.308939, -111.049764
\item AGL: 90 and 100 meters
\item Camera: DJI FC6310S
\item Ground Sampling Distance: 2.62 cm\footnote{This is the value from the Pix4D quality report, as the flights are at two different altitudes}
\end{itemize}


\begin{figure}[H]
\centering
  \includegraphics[width=.5\linewidth]{figures/flight-paths.png} 
  \caption{Flight plan for study area}
  \label{fig:flight} 
\end{figure}

Figure~\ref{fig:flight} shows the flight paths executed for this flight, and a few problems are immediately obvious on closer inspection. While the intent was to have the entire study area covered in a double-grid pattern, a double grid pattern was used in approximately $\frac {1} {3}$ of the study area. Unfortunately, the flight control system crashed after only completing a small area, requiring the use of a second flight control system where a double-grid pattern was not applied. Subsequent processing of the imagery acquired is the result of combining these two image sets, with $\frac {1} {3}$ of the flight at 100 meters giving a ground sampling distance of 2.74 cm and $\frac {2} {3}$ of the flight at 90 meters, giving a ground sampling distance of 2.47 cm. These two data sets were combined, giving the GSD shown. The sky was clear with minimal wind during the flights. A nearby fire that produced large amounts of smoke did not affect the flight area with obscuring haze.

The images were initially processed with Pix4D to produce an initial point cloud and orthomosaic. Subsequent processing was done with software written by the author and with verification with R (version 4.1.1) and RStudio (version 2021.09.0 Build 351). (Both Python and R sources are available in the GitHub repository referenced in the appendix)

The workflow follows this sequence:
\begin{enumerate}
	\item Acquire Drone Images
	\item Upload Image Sets to Cyverse (or equivalent)
	\item Produce Point Cloud and Orthomosaic
	\item Produce segmented image from orthomosaic
	\item Calculate vegetation area from segmented area
	\item Calculate vegetation volume from point cloud data
\end{enumerate}

%
% I M A G E  S E G M E N T A T I O N
%
\section{Image Segmentation}
The portions of the images that did not contain pixels with vegetation present were discarded. These images were segmented using various visible light indices \cite{Hamuda2016-dw}. As this process is not the primary subject of this paper, it will be given only superficial mention here.  Various approaches to image segmentation are  summarized in Table \ref{fig:indices}. These indices were coded

%
% T A B L E  O F  S E G M E N T A T I O N  F O R M U L A E
%
% A bit more space on the table rows, so everything does not look so cramped
{\renewcommand{\arraystretch}{2}%
\begin{table}[H]
%\begin{longtable}{ccc}
    \caption{Visible light indices (\cite{Hamuda2016-dw})}
    \label{fig:indices}
    \begin{tabular}{  l  p{4cm}  p{5cm} }
        \toprule
\textbf{Index}      
& \textbf{Formula}   
& \textbf{Comment} \\\midrule
Triangular Greenness
& \begin{minipage}[t]{0.3\textwidth}
	$R_{green} - \alpha R_{red} - \beta R_{blue}\\ \alpha = \frac {2(\lambda_{blue} - \lambda_{green})} {(\lambda_{blue} - \lambda_{red})}\\ 
    	\beta = \frac {2(\lambda_{green} - \lambda_{red})} {(\lambda_{blue} - \lambda_{red})} $
   \end{minipage}     
& Corrects for camera calibration using the peak sensitivity \\\hline
Normalized Difference     
& $128 * \left( \left( \frac {(G - R)} {(G + R)} \right) + 1 \right) $                    
& The NDI index produces a near-binary image.  \\\hline
Excess Green      
& \begin{minipage}[t]{0.3\textwidth}
	$R = \frac {R} {R_{max}}\\ G = \frac {G} {G_{max}}\\ B = \frac {B} {B_{max}}$ 
   \end{minipage}
& ExG provided a clear contrast between plants and soil \\\hline
Excess Red      
& $1.3 R - G$ 
& inspired by the fact that there are 4\% blue, and 32\% green, compared with 64\% red cones in the retina of the human eye \\\hline
Color Index of Vegetation Extraction      
& $0.441 R - 0.811 G + 0.385 B + 18.78745$
& This method was proposed to separate green plants from soil background in order to evaluate the crop growing status. \\\hline
Excess Green - Excess Red   
& $ExG - ExR$ 
& ExG used to extract the plant region and ExR used to eliminate the background noise (soil and residue) where green–red material (stems, branches, or petioles) may exist \\\hline
Normalized Green-Red Difference    
& $\frac {(G - R)} {(G + R)}$ 
& The method of NGRDI was used to overcome the differences in exposure settings selected by the digital camera when acquiring aerial photography of the field. \\\hline
Vegetative Index      
& $\frac {G} {R^aB^{(1-a)}}, a = 0.667$ 
& VEG has a significant advantage because it is robust to lighting change.\\\hline
Com1   
& $ExG + CIVE + ExGR + VEG$ 
& TODO \\\hline
Modified Excess Green      
& $1.262G - 0.884R = 0.311B$ 
& TODO \\\hline
Combined Indices 2      
& $0.36ExG + 0.47CIVE + 0.17VEG$ 
& Uses weighting factors to emphasize strengths of various approaches\\\hline    

        \bottomrule
    \end{tabular}
\end{table}
%\end{longtable}
These indices are used to create a mask that is then applied to the original source image to permit vegetation to show while masking details that are not relevant (ground pixels, stones, and other items that may appear in field conditions) The intent here is to remove all pixels that are not relavent to the task of distinguishing between ground and ground features and vegetation.  That is, in Figure~\ref{fig:sub2}, the black areas indicate non-vegetated pixels (ground, rock, etc., anything other than vegetation) and pixels showing green are vegetation.
\begin{figure}[H]
\centering
\begin{subfigure}[]{.30\textwidth}
  \centering
  % TIFF format is something latex can't handle, so this is manually converted to JPG with photoshop.
  \includegraphics[width=1\linewidth]{figures/final_transparent_mosaic_group1.jpg} 
  \caption{Field view of study area}
  \label{fig:sub1}
\end{subfigure}
\begin{subfigure}[]{.30\textwidth}
  \centering
  \includegraphics[width=1\linewidth]{figures/index.jpg} 
  \caption{CIVE index}
  \label{fig:sub2}
\end{subfigure}
\begin{subfigure}{.30\textwidth}
  \centering
  \includegraphics[width=1\linewidth]{figures/final_transparent-mosiac_group1-cive.jpg}
  \caption{After segmentation using CIVE}
  \label{fig:sub2}
\end{subfigure}
\caption{Image Processing with Vegetation Index CIVE}
\label{fig:segmentation}
\end{figure}
The segmented image has discarded ground pixels while retaining most of the pixels that will be used, but a close examination reveals that pixels of ground areas where vegetation is present experience some color bleed from adjacent pixels that are vegetated. With any segmentation process, however, it is important to select the {\it threshold} such that results can be reproduced. A threshold is used in conjunction with an index such that only those pixels greater than a certain value in the mask will be applied. A common approach is to use Otsu's method \cite{Otsu1979-io} , as manual methods are typically hit-or-miss, however this paper will use manually computed thresholds for reasons that will be discussed in a subsequent section.
\begin{figure}[H]
  \centering
  \begin{subfigure}{.32\textwidth}
  	\centering
  	\includegraphics[width=1\linewidth]{figures/exgexr.jpg}
  	\caption{After segmentation using ExGExR}
  	\label{fig:exgexr-overview}
  \end{subfigure}
  \begin{subfigure}{.32\textwidth}
  	\centering
  	\includegraphics[width=1\linewidth]{figures/detail-of-river.jpg}
  	\caption{Detail of this area}
  	\label{fig:exgexr-detail}
  \end{subfigure}
  \caption{Use of Excessive Green - Excessive Red Index}
  \label{fig:exgexr}
\end{figure}

Figure~\ref{fig:exgexr-overview} shows more water than seen in Figure~\ref{fig:sub2}, illustrating the importance of careful threshold selection as well as selection of a segmentation technique.  Figure~\ref{fig:index-plot-3d} shows a subset of the study area and illustrates how thresholds are applied to a generated index. This shows a subset of the study area to frame the concept of a threshold. In this plot we can see two distinct groups: vegetation, noted by green, and items other than vegetation, noted by brown. A good starting point for estimating the threshold in this instance is using a value of 120.

\begin{figure}[H]
\centering
    \begin{subfigure}{.40\textwidth}
  	\centering
  	\includegraphics[width=1\linewidth]{figures/3d-plot-ndi.jpg}
  	\caption{Index plot for threshold estimation}
  	\label{fig:exgexr-3d-oblique}
  \end{subfigure}
  \begin{subfigure}{.40\textwidth}
  	\centering
  	\includegraphics[width=1\linewidth]{figures/3d-plot-rotated-ndi.jpg}
  	\caption{Overhead view of plot}
  	\label{fig:exgexr-overhead}
  \end{subfigure} 
  \caption{Scatterplots of NDI Index}
  \label{fig:index-plot-3d}
\end{figure}

Starting with the threshold of 120 and observing the effects of various values does not have the effect desired in this instance, as the river is still shown.  A transect through a region across the river reveals that the river reflects wavelengths in the band we have chosen as the threshold, but absorbs wavelengths below that threshold.  (See the area from location 1000 to about 1200)

\begin{figure}[H]
\centering
  	\includegraphics[width=1\linewidth]{figures/transect-plot.png}
  	\label{fig:exgexr}
  \caption{Transect across river using Normalized Difference}
    \label{fig:transect-plot} 
\end{figure}

The river is a bit confounding here, but now that the radiometric response is known, it can be eliminated by exploiting a characteristic seen in Figure\~ref{fig:transect-plot}: band separation in the area of the river.  Note how the red, greed, and blue bands have much greater separation over the river than the areas over trees.  This process is repeated for each index that reveals not only vegetation, but water as well. Not all indices suffer from this, as seen in Figure~\ref{fig:transect-plot-exg}, showing a transect across the river at the same location.  The transect shows that the river cannot be seen in the image.

\begin{figure}[H]
\centering
  \includegraphics[width=1\linewidth]{figures/transect-plot-exg.png}
  \caption{Transect across river using Excess Green index}
  \label{fig:transect-plot-exg} 
\end{figure}

Figure~\ref{fig:barcount} shows the results of using various indices to determine vegetated area in the orthomosaic of the study area.  While it is tempting to asess these results by choosing the one with the most vegetated area, the emphasis here will be on choosing the most accurate results. Unfortunately, this is where careful comparisons must be made between the image with the index applied and the original to assess if vegetation has been wrongfully excluded, or as discussed above wrongfully included (such as we see with the river).

\begin{figure}[H]
\centering
  	\includegraphics[width=1\linewidth]{figures/bar-counts.jpg}
  	 \caption{Areas calculated for various algorithms}
  	\label{fig:barcount}
\end{figure}


\subsection{Implementation Notes}
Most of implementation of these index algorithms are provided by the author, with these supporting Python libraries:
\begin{itemize}
\item{{\it OpenCV}, an image processing subsystem}
\item{{\it numpy}, a numerical processing package}
\end{itemize}


%
% P O I N T  C L O U D S
%
\section{Segmentation of Point Clouds}
Vegetation indices can also be applied to point clouds with a few items of note:
\begin{itemize}
\item{The values for the RGB points may be substatially different than those seen in the orthomosaic In the example shown here, the threshold used with the index was 5000 -- contrasted with the value of 130.}
\item{Automatic selection of the threshold is a bit more complicated, as the technique mentioned (Otsu) involved a grey-scale image, something not present here}
\item{Point clouds involve many more points than an orthomosaic for the same study area, and processing time will be substatially longer}
\end{itemize}
Figure~\ref{fig:point-cloud-mexg} shows the effect of applying the {\it Modified Excess Green} index to a point cloud of the study area.

\begin{figure}[H]
\centering
  	\includegraphics[width=1\linewidth]{figures/point-cloud-mexg.jpg}
  	 \caption{Modified Excess Green Index applied to Point Cloud}
  	\label{fig:point-cloud-mexg}
\end{figure}

\subsection{Implementation Notes}
This implementation builds on the algorithms realized in the section detailing 2D work, and uses some supporting libraries:
\begin{itemize}
\item{{\it laspy}, an subsystem for manipulating point clouds in the {\it .las} format}
\item{{\it numpy}, a numerical processing package}
\end{itemize}

It is worth mentioning here that processing the study area point cloud could not be done completely with the numpy numeric functions used in the 2D processing. In this case, numpy required 995 GB for processing, requiring creation of some new routines. A point cloud in the {\it .las} format is nothing more complicated than a long list of all points in the cloud
%
% R E S U L T S  & D I S C U S S I O N
%
\section{Discussion of Results}


% A bit more space on the table rows, so everything does not look so cramped
{\renewcommand{\arraystretch}{2}%
\begin{table}[H]
%\begin{longtable}{ccc}
    \caption{Calculations of Vegetation Area Using Various Segmentation Approaches}
    \label{fig:calculations}
    \begin{tabular}{  l  p{4cm}  p{5cm} }
        \toprule
\textbf{Index}      
& \textbf{Area}   
& \textbf{Comment} \\\midrule
Triangular Greenness
&  2525104
& \\\hline
Normalized Difference     
&   2806857
&   Picks up too much of the bright areas. (includes river)\\\hline
Excess Green      
& 2452001
& \\\hline
Excess Red      
& 2962501
&  Tends to pick up too much redness in ground. (includes river)\\\hline
Color Index of Vegetation Extraction      
& 497783
& Does not include river\\\hline
Excess Green - Excess Red   
& 2411091
& Includes river \\\hline
Normalized Green-Red Difference    
& 3345184
& Includes River\\\hline
Vegetative Index      
& 1004538
& Does not include river\\\hline
Modified Excess Green      
& 1331686
& Does not include river\\\hline
Combined Indices 1  
&  131424
& Does not pick up all the vegetation (does not include river)\\\hline    
Combined Indices 2      
&  1543544
& Picks up even light colored vegetation, even unhealthy (does not include river)\\\hline    

        \bottomrule
    \end{tabular}
\end{table}

\begin{figure}[H]
\centering
\begin{subfigure}{.32\textwidth}
	\centering
	\includegraphics[width=1\linewidth]{figures/processed-with-com2.jpg}
	\caption{COM2}
\end{subfigure}
\begin{subfigure}{.32\textwidth}
	\centering
	\includegraphics[width=1\linewidth]{figures/processed-with-exg.jpg}
	\caption{Excess Green}
\end{subfigure}
\begin{subfigure}{.32\textwidth}
	\centering
	\includegraphics[width=1\linewidth]{figures/processed-with-cive.jpg}
	\caption{CIVE}
\end{subfigure}
\begin{subfigure}{.32\textwidth}
	\centering
	\includegraphics[width=1\linewidth]{figures/processed-with-com1.jpg}
	\caption{COM1}
\end{subfigure}
\begin{subfigure}{.32\textwidth}
	\centering
	\includegraphics[width=1\linewidth]{figures/processed-with-exgexr.jpg}
	\caption{ExgExr}
\end{subfigure}
\begin{subfigure}{.32\textwidth}
	\centering
	\includegraphics[width=1\linewidth]{figures/processed-with-exr.jpg}
	\caption{Excess Red}
\end{subfigure}
\begin{subfigure}{.32\textwidth}
	\centering
	\includegraphics[width=1\linewidth]{figures/processed-with-mexg.jpg}
	\caption{Modified Excess Green}
\end{subfigure}
\begin{subfigure}{.32\textwidth}
	\centering
	\includegraphics[width=1\linewidth]{figures/processed-with-ndi.jpg}
	\caption{NDI}
\end{subfigure}
\begin{subfigure}{.32\textwidth}
	\centering
	\includegraphics[width=1\linewidth]{figures/processed-with-ngrdi.jpg}
	\caption{NGRDI}
\end{subfigure}
\begin{subfigure}{.32\textwidth}
	\centering
	\includegraphics[width=1\linewidth]{figures/processed-with-tgi.jpg}
	\caption{TGI}
\end{subfigure}
\begin{subfigure}{.32\textwidth}
	\centering
	\includegraphics[width=1\linewidth]{figures/processed-with-veg.jpg}
	\caption{VEG}
\end{subfigure}
\caption{Results of processing the study area with various indices}
\end{figure}

\section{Accuracy Assessment}
Systematically assessing the accuracy of the beyond the scope of this paper, but discussed extensively elsewhere \cite{Barati2011-oj}. Overlaying the vegetated areas revealed through the indices (depicted in red) with the original images reveals Figure~\ref{fig:overlays}. In this subset, we can compare the original images with the segmentation approaches.

\subsection{CIVE}
The CIVE index does not capture portions of the same plant, leading to a large underestimation in plant area.
\begin{figure}[H]
	\centering
	\includegraphics[height=5cm]{figures/detail-cive-overlay.jpg}
	\caption{Detail of the CIVE overlay}
	\label{fig:detail-cive}
\end{figure}

\subsection{TGI}
The TGI index does a good job of detecting vegetation within the plant, and does not exhibit false positives within shadow areas. As Figure~\ref{fig:detail-tgi} shows, the shadow above the isolated tree does not indicate that it was counted as vegetation. However, this index does exlude vegetation that is lower green but additional ground work is required to determine it if is living (the vegetation in the top right of the image).
\begin{figure}[H]
	\centering
	\includegraphics[height=5cm]{figures/detail-tgi-overlay.jpg}
	\caption{Detail of the TGI overlay}
	\label{fig:detail-tgi}
\end{figure}

\subsection{COM2}
The COM2 index does a good job of detecting vegetation within the plant, and does not exhibit false positives within shadow areas. As Figure~\ref{fig:detail-tgi} shows, however, this index does not capture portions of the same plant.
\begin{figure}[H]
	\centering
	\includegraphics[height=5cm]{figures/detail-com2-overlay.jpg}
	\caption{Detail of the COM2 overlay}
	\label{fig:detail-com2}
\end{figure}

\subsection{Excessive Green}
The ExG index does a good job of detecting vegetation within the plant, and does not exhibit false positives within shadow areas. As Figure~\ref{fig:detail-exg} shows, however, this index tends to include bright sand in the midst of trees. It is possible that this is a reflection of the tree reflectance, but bears further investigation. Other bright sandy areas are not incorrectly marked as vegetation.
\begin{figure}[H]
	\centering
	\includegraphics[height=5cm]{figures/detail-exg-overlay.jpg}
	\caption{Detail of the Excessive Green overlay}
	\label{fig:detail-exg}
\end{figure}

\subsection{Excessive Green Excessive Red}
The ExGExR index does a good job of detecting vegetation within the plant, and does not exhibit false positives within shadow areas. As Figure~\ref{fig:detail-exgexr} shows, however, this index fails to include vegetation with a light hue.
\begin{figure}[H]
	\centering
	\includegraphics[height=5cm]{figures/detail-exgexr-overlay.jpg}
	\caption{Detail of the Excessive Green Excessive Red overlay}
	\label{fig:detail-exgexr}
\end{figure}

\subsection{Excessive Red}
Overall, this index captures much of the vegetation, but as Figure~\ref{fig:detail-exr} shows, this index fails to include vegetation that is fully green. While the detail shown in this picture is well beyond the range of the river, it seems consistently colored.
\begin{figure}[H]
	\centering
	\includegraphics[height=5cm]{figures/detail-exr-overlay.jpg}
	\caption{Detail of the Excessive Red overlay}
	\label{fig:detail-exr}
\end{figure}

\subsection{Modified Excess Green}
It is useful here to contrast this index with the results seen in the same area, As Figure~\ref{fig:detail-mexg} shows, this index does capture vegetation that is fully green, but then fails to capture vegetation around it (likely due to the red band content of the vegetation.
\begin{figure}[H]
	\centering
	\includegraphics[height=5cm]{figures/detail-mexg-overlay.jpg}
	\caption{Detail of the Modified Excessive Green overlay}
	\label{fig:detail-mexg}
\end{figure}

\subsection{Normalized Difference}
The NDI index captures most of the vegetation with few exceptions. As Figure~\ref{fig:detail-ndi} shows, this index tends to fail to capture lightly colored vegetation.
\begin{figure}[H]
	\centering
	\includegraphics[height=5cm]{figures/detail-ndi-overlay.jpg}
	\caption{Detail of the Normalized Difference overlay}
	\label{fig:detail-ndi}
\end{figure}

\subsection{Normalized Green Red Difference}
The NGRDI index captures most of the vegetation but tends to also capture portions of the image without vegetation. As Figure~\ref{fig:detail-ngrdi} shows, this index captures brightly colored sand areas devoid of vegetation, distorting calculations of vegetation coverage.
\begin{figure}[H]
	\centering
	\includegraphics[height=5cm]{figures/detail-ngrdi-overlay.jpg}
	\caption{Detail of the Normalized Green Red Difference overlay}
	\label{fig:detail-ngrdi}
\end{figure}

\subsection{Triangular Greenness}
The TGI index captures most of the vegetation with few problems seen. Examining the area other indices did not perform satisfactorily  Figure~\ref{fig:detail-tgi} shows that the TGI index captured most vegetation in the area. 
\begin{figure}[H]
	\centering
	\includegraphics[height=5cm]{figures/detail-tgi-overlay.jpg}
	\caption{Detail of the Triangular Greenness overlay}
	\label{fig:detail-tgi}
\end{figure}

\subsection{Vegetative}
The Vegetative index failed to capture large amounts of vegetation, as Figure~\ref{fig:detail-veg} reveals 
\begin{figure}[H]
	\centering
	\includegraphics[height=5cm]{figures/detail-veg-overlay.jpg}
	\caption{Detail of the Vegetative overlay}
	\label{fig:detail-veg}
\end{figure}

\begin{figure}[H]
\centering
\begin{subfigure}{.32\textwidth}
	\centering
	\includegraphics[width=1\linewidth]{figures/contours-for-com2.jpg}
	\caption{COM2}
\end{subfigure}
\begin{subfigure}{.32\textwidth}
	\centering
	\includegraphics[width=1\linewidth]{figures/contours-for-exg.jpg}
	\caption{Excess Green}
\end{subfigure}
\begin{subfigure}{.32\textwidth}
	\centering
	\includegraphics[width=1\linewidth]{figures/contours-for-cive.jpg}
	\caption{CIVE}
\end{subfigure}
\begin{subfigure}{.32\textwidth}
	\centering
	\includegraphics[width=1\linewidth]{figures/contours-for-com1.jpg}
	\caption{COM1}
\end{subfigure}
\begin{subfigure}{.32\textwidth}
	\centering
	\includegraphics[width=1\linewidth]{figures/contours-for-exgexr.jpg}
	\caption{ExgExr}
\end{subfigure}
\begin{subfigure}{.32\textwidth}
	\centering
	\includegraphics[width=1\linewidth]{figures/contours-for-exr.jpg}
	\caption{Excess Red}
\end{subfigure}
\begin{subfigure}{.32\textwidth}
	\centering
	\includegraphics[width=1\linewidth]{figures/contours-for-mexg.jpg}
	\caption{Modified Excess Green}
\end{subfigure}
\begin{subfigure}{.32\textwidth}
	\centering
	\includegraphics[width=1\linewidth]{figures/contours-for-ndi.jpg}
	\caption{NDI}
\end{subfigure}
\begin{subfigure}{.32\textwidth}
	\centering
	\includegraphics[width=1\linewidth]{figures/contours-for-ngrdi.jpg}
	\caption{NGRDI}
\end{subfigure}
\begin{subfigure}{.32\textwidth}
	\centering
	\includegraphics[width=1\linewidth]{figures/contours-for-tgi.jpg}
	\caption{TGI}
\end{subfigure}
\begin{subfigure}{.32\textwidth}
	\centering
	\includegraphics[width=1\linewidth]{figures/contours-for-veg.jpg}
	\caption{VEG}
\end{subfigure}
\caption{Shaded overlays of study area with various indices. Red areas indicate vegetation.}
\label{fig:overlays}
\end{figure}

%
% C O N C L U S I O N S
%
\section{Conclusions}
In this paper, we have explored using the application of vegetation indices to both 2D and 3D data as a part of a workflow to estimate vegetation area and volume. Of course, this is only a snapshot of one data. The data acquisition  and subsequent processing must be repeated across seasons to evaluate vegetation coverage.  The methods here, however, will suffer from a few items:
\begin{itemize}
\item{Lighting conditions will vary between data sets acquired on different dates. It may be the case that some indices are sensitive to these effects and will yield different results even when processing the same area}
\item{Manual selection of a threshold point across seasons may yield misleading results.  The selection of the threshold for2D data can be automated, but }
\end{itemize}


%
% R E F E R E N C E S
%
\section{References}
\printbibliography[heading=none]

% 
% A P P E N D I X
%
\section{Access to Data and Code}
All code written for this project is available from this Github repository:\\
\url{https://github.com/evan-mcginnis/weeds/tree/rnr522}




\end{document}

